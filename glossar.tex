% coding:utf-8

%FOSASTOC, a LaTeX-Code for a electrical summary of stochastic
%Copyright (C) 2013, Daniel Winz, Ervin Mazlagic

%This program is free software; you can redistribute it and/or
%modify it under the terms of the GNU General Public License
%as published by the Free Software Foundation; either version 2
%of the License, or (at your option) any later version.

%This program is distributed in the hope that it will be useful,
%but WITHOUT ANY WARRANTY; without even the implied warranty of
%MERCHANTABILITY or FITNESS FOR A PARTICULAR PURPOSE.  See the
%GNU General Public License for more details.
%----------------------------------------

\newglossaryentry{arithmetische Mittel}{
	name={Arithmetisches Mittel},
	text={arithmetische Mittel},
	description={Quotient aus der Summe aller Elemente und
		Anzahl Elemente},
	sort={Arithmetisches Mittel}
}

\newglossaryentry{Mittel}{
	name=Mittel,
	description={Mittel ist eine Kurzform für das 
		\emph{arithmetische Mittel}},
	sort=Mittel
}

\newglossaryentry{Durchschnitt}{
	name=Durchschnitt,
	description={Durchschnitt ist eine Kurzform für das
		\emph{arithmetische Mittel}},
	sort=Durchschnitt
}

\newglossaryentry{Quantil}{
	name=Quantil,
	description={Prozentuale Grenze innerhalb einer Datenreihe},
	sort=Quantil
}

\newglossaryentry{Quartil}{
	name=Quartil,
	description={Ein spezielles Quantil welches ein Vielfaches
		vom $25\%$-Quantil ist},
	sort=Quartil
}

\newglossaryentry{Median}{
	name=Median,
	description={Das $50\%$-Quantil},
	sort=Median
}

\newglossaryentry{Varianz}{
	name=Varianz,
	description={Mass der (quadratischen) Abweichung von Datenpunkten 
		und artihmetischem Mittel einer Datenreihe, d.h. das
		Quadrat der Standardabweichung},
	sort=Varianz
}

\newglossaryentry{Standardabweichung}{
	name=Standardabweichung,
	description={Mass der (linearen) Abweichung von Dantenpunkten
		und arithmetischem Mittel einer Datenreihe, d.h. die
		Quadratwurzel der Varianz},
	sort=Standardabweichung
}

\newglossaryentry{Kovarianz}{
	name=Kovarianz,
	description={Mass der Ähnlichkeit von Datenreihen},
	sort=Kovarianz
}

\newglossaryentry{Korrelation}{
	name=Korrelation,
	description={Zusammenhang verschiedener Variablen},
	sort=Korrelation
}

\newglossaryentry{Korrelationskoeffizient}{
	name=Korrelationskoeffizient,
	description={Ein (prozentuales) Mass für den Zusammenhang 
		verschiedener Variablen},
	sort=Korrelationskoeffizient
}

\newglossaryentry{Mengenlehre}{
	name=Mengenlehre,
	description={Ein Teilgebiet der Mathematik welches aus der
		Logik hervorgeht. In der Stochastik findet die
		Mengenlehre insbesondere in der Kombinatorik 
		Verwendung},
	sort=Mengenlehre
}

\newglossaryentry{Abstraktion}{
	name=Abstraktion,
	description={Vereinfachung auf Wesentliches}
}

\newglossaryentry{Ereignisraum}{
	name=Ereignisraum,
	description={Die Menge aller möglichen Ereignisse (oft mit
		$\Omega$ notiert)},
	sort=Ereignisraum
}

\newglossaryentry{Teilmenge}{
	name=Teilmenge,
	description={Menge von Elementen die alle Teil einer anderen
		Menge sind},
	sort=Teilmenge
}

\newglossaryentry{echte Teilmenge}{
	name={Echte Teilmenge},
	text={echte Teilmenge},
	description={Menge von Elementen die alle Teil einer anderen
		Menge sind, diese aber noch weitere Elemente enthält,
		d.h. die Mengen sind nicht gleich},
	sort={Echte Teilmenge}
}

\newglossaryentry{leere Menge}{
	name={Leere Menge},
	text={leere Menge},
	description={Eine Menge die keinerlei Elemente enthält},
	sort={Leere Menge}
}

\newglossaryentry{Schnittmenge}{
	name=Schnittmenge,
	description={Menge von Elementen, die alle in mehreren Mengen
		enthalten sind},
	sort=Schnittmenge
}

\newglossaryentry{Vereinigungsmenge}{
	name=Vereinigungsmenge,
	description={Menge der Elemente die in mindestens einer anderen
		Menge vorkommen},
	sort=Vereinigungsmenge
}

\newglossaryentry{gleiche Menge}{
	name={Gleiche Menge},
	text={gleiche Menge},
	description={Eine Menge deren Inhalt identisch ist mit dem
		Inhalt einer anderen Menge},
	sort={Gleiche Menge}
}

\newglossaryentry{Differenzmenge}{
	name=Differenzmenge,
	description={Menge von Elementen einer Menge ohne jene Elemente,
		welche auch in einer weiteren Menge enthalten sind},
	sort=Differenzmenge
}

\newglossaryentry{Kolmogorov}{
	name=Kolmogorov,
	description={Bekannter russischer Mathemathiker, welche die
		Grundlegenden Axiome der Wahrscheinlichkeitstheorie
		formulierte (siehe \gls{Axiome von Kolmogorov})},
	sort=Kolmogorov
}

\newglossaryentry{Axiome von Kolmogorov}{
	name={Axiome von Kolmogorov},
	description={Der russische Mathematiker Andrei N. Kolmogorov
		formulierte die drei 
		grundlegenden Axiome der Wahrscheinlichkeitstheorie;
		Nichtnegativität, Normiertheit, Additivität},
	sort={Axiome von Kolmogorov}
}

\newglossaryentry{Nichtnegativitaet}{
	name={Nichtnegativität},
	text={Nichtnegativität},
	description={Bezeichnung für das erste Axiom von Kolmogorov.
		Dieses besagt, dass es [\dots] keine negativen 
		Wahrscheinlichkeiten gibt},
	sort={Nichtnegativitaet}
}

\newglossaryentry{Normiertheit}{
	name=Normiertheit,
	description={Bezeichnung für das zweite Axiom von Kolmogorov.
		Dieses besagt, dass der Grundraum $\Omega$ eine
		Wahrscheinlichkeit von $P(\Omega)=1$ hat und sich
		stets etwas ereignet ($P(\emptyset)=0$)},
	sort={Normiertheit}
}

\newglossaryentry{Additivitaet}{
	name={Additivität},
	text={Additivität},
	description={Bezeichnung für das zweite Axiom von Kolmogorov.
		Dieses besagt, dass die Wahrscheinlichkeiten sich
		ausschliessender Ereignisse addieren lassen},
	sort={Additivitaet}
}

\newglossaryentry{stochastisch unabhaengig}{
	name={Stochastische Unabhängigkeit},
	text={stochastisch unabhängig},
	description={Ereignisse welche sich gegenseitig nicht
		beeinflussen nennt man stochastisch unabhängig},
	sort={Stochastisch Unabhaengig}
}

\newglossaryentry{Potenzmenge}{
	name=Potenzmenge,
	description={Eine Menge aus allen Teilmengen einer anderen 
		Menge},
	sort=Potenzmenge
}

\newglossaryentry{Komplement}{
	name=Komplement,
	description={Differenzmenge einer Potenzmenge und einer Menge
		aus derselben Potenzenge},
	sort=Komplement
}

\newglossaryentry{Beyes'sche Theorem}{
	name={Beyes'sche Theorem},
	description={Das Beye'sche Theorem ist vom Mathematiker
		Thomas Beyes formulierter Satz, welcher die Berechnung
		bedingter Wahrscheinlichkeiten beschreibt},
	sort={Beyes'sche Theorem}
}
