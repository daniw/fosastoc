% coding:utf-8

%FOSASTOC, a LaTeX-Code for a electrical summary of stochastic
%Copyright (C) 2013, Daniel Winz, Ervin Mazlagic

%This program is free software; you can redistribute it and/or
%modify it under the terms of the GNU General Public License
%as published by the Free Software Foundation; either version 2
%of the License, or (at your option) any later version.

%This program is distributed in the hope that it will be useful,
%but WITHOUT ANY WARRANTY; without even the implied warranty of
%MERCHANTABILITY or FITNESS FOR A PARTICULAR PURPOSE.  See the
%GNU General Public License for more details.
%----------------------------------------

\chapter{Grundbegriffe}
\newpage

\section{Arithmetisches Mittel}
Das arithmetische Mittel beschreibt den Quotienten aus 
der Summe aller Elemente und der Anzhal Elemente. Dieses wird oft auch
einfach als Mittelwert oder Durchschnitt bezeichnet. Ein solcher
Mittelwert wird mittels eines übergestellten Striches markiert z.B.
$\bar{x}$.
\[ 
	\bar{x} 
	= \frac{1}{n} \cdot \left( \sum_{i=1}^{n} x_i \right)
	= \frac{x_1 + x_2 + \dots + x_n}{n}
\]
Ein Alltagsbeispiel für das arithmetische Mittel ist der 
\emph{Notendurchschnitt}. 

In \lstinline{R} wird das arithmetische Mittel mit 
\lstinline{mean()} ermittelt.
\begin{Schunk}
\begin{Sinput}
> a <- c(6,4,5,5,4,5,6,3,4,5,6)
> mean(a)
\end{Sinput}
\begin{Soutput}
[1] 4.818182
\end{Soutput}
\end{Schunk}

\section{Quantil}
Ein Quantil beschreibt eine Grenze innerhalb einer sortierten 
Datenreihe. Diese wird stets in Prozenten angegeben, d.h. 
\emph{5\% Qunatil, 20\% Quantil} usw. 
Beispielsweise ein 5\% Quantil ist eine Stelle in der Datenreihe, an 
welcher 5\% der Werte unterhalb dieser Grenze liegen (z.B. \emph{5\% 
aller E-Mails sind kürzer als 12 Worte $\Rightarrow$ 12 ist das 5\%
Quantil}).
\[ \begin{array}{l l}
	\alpha \in [0, 1] 
		& \text{Prozentwert der Grenze} \\
	x_1 - x_n
		& \text{Elemente sortiert nach grösse} \\
	\alpha \cdot n 
		& \text{Qunatil} \\
\end{array} \]
Bei der Berechnung des Quantils gilt es zu beachten ob die Elemente 
der Datenreihe ganze oder gebrochene Zahlen beinhalten, denn dies 
beeinflusst die Rechnung.
\[ \begin{array}{l l}
	\text{ganze Zahlen:}
		& \frac{1}{2} \cdot \left(
			x_{\alpha \cdot n} 
			+ x_{\alpha \cdot (n + 1)} \right)  \\
	& \\
	\text{gebrochene Zahlen:}
		& k = \alpha \cdot n + \frac{1}{2}  \\
		& k \text{ runden} \\
		& \Rightarrow x_{k}
\end{array} \]
Die Brechnung von Qunatilen in \lstinline{R} lässt sich mittels 
\lstinline{quantile()} durchführen. Hierbei gilt es zu beachten, dass
\lstinline{R} neun verschiedene Algorithmen kennt um Quantile zu 
berechnen. Ohne einen Parameter nimmt \lstinline{R} den 
Default-Algorithmus (\lstinline{type=7}). Für sämtliche Aufgaben aus dem
Stochastikunterricht der HSLU gilt der \lstinline{type=2} als einzig
richtiger! 

Im Folgenden eine Beispielsrechnung vom 30\% Quantil 10 zufälliger 
ganzer Zahlen zwischen 1 und 20.
\begin{Schunk}
\begin{Sinput}
> x <- round(x=runif(n=10, min=1, max=20), digits=0)
> x <- sort(x)
> x
\end{Sinput}
\begin{Soutput}
 [1]  1  2  3  6  7  7  9  9 18 19
\end{Soutput}
\begin{Sinput}
> quantile(x, prob=0.2, type=2)
\end{Sinput}
\begin{Soutput}
20% 
2.5 
\end{Soutput}
\end{Schunk}

\section{Quartil}
Das Quartil beschreibt spezielle Quantile welche jeweils ein
Vielfaches von $\frac{1}{4}$ bzw. $25\%$ sind, d.h. das
$25\%$, $50\%$ als auch das $75\%$ Quantil.

\section{Median}
Der Median ist wie das Quartil ein spezielles Quantil, welches im Falle
des Medians das $50\%$ Quantil beschreibt.

In der Statistik hat der Median eine wichtige Bedeutung, denn er 
beschreibt lediglich das mittlere Element einer sortierten Datenreihe.
Somit ist der Median unenpfindlich gegenüber Extremwerten (anders das
arithmetische Mittel).

In \lstinline{R} lässt sich der Median mittels \lstinline{median()}
ermitteln. Im Folgenden ein Beispiel welches verdeutlicht, dass der
Median umempfindlich ist gegen Extremwerte bzw. Ausreisser.
\begin{Schunk}
\begin{Sinput}
> x <- c(1.6, 1.7, 1.75, 1.87, 1.94)
> y <- c(1.2, 1.4, 1.75, 1.99, 2.17)
> # sort() wird ausgelassen, da bereits sortiert
> median(x); median(y)
\end{Sinput}
\begin{Soutput}
[1] 1.75
\end{Soutput}
\begin{Soutput}
[1] 1.75
\end{Soutput}
\end{Schunk}

\section{Varianz}
Die Varianz beschreibt die quadratische Abweichung von Daten und
ihrem arithmetischen Mittelwert. Sie ist somit das Quadrat der 
Standardabweichung und wird deshalb als $\sigma^2$, $\sigma_{x}^2$ 
bzw. $s_{x}^2$ notiert.
\[
	\sigma_{x}^2 
	= s_{x}^2 
	= \frac{1}{n-1} \cdot \left(
		\sum_{i=1}^n (x_i-\bar{x})^2 
	\right)
\]
Der lineare Faktor $\frac{1}{n-1}$ ist nicht intutiv aber dennoch 
korrekt(er) aufgrund einer systematischen Abweichung bei der 
Berechnung mit $\frac{1}{n}$.

\section{Standardabweichung}
Die Standardabweichung beschreibt wie gross die die mittlere Abweichung
ist von den einzelnen Beobachtungen (Datenpunkten) zum arithmetischen
Mittel derselben Beobachtungen. Die Standardabweichung ist somit gleich
der zweiten Wurzel von der Varianz den selben Beobachtungen. Die
Standardabweichung wird oft mit $\sigma$, $\sigma_x$ bzw. $s$, $s_x$ 
notiert.
\[
	\sigma_x 
	= s_x 
	= \sqrt{\sigma_{x}^2}
	= \sqrt{s_{x}^2}
	= \sqrt{\frac{1}{n-1} \cdot \left(
		\sum_{i=1}^n (x_i-\bar{x})^2 
	\right)}
\]
In \lstinline{R} wird die Standardabweichung mittel \lstinline{sd()}
ermittelt. Im Folgenden ein Beispiel welches die Standardabweichung
10 zufälliger ganzer Zahlen ermittelt.
\begin{Schunk}
\begin{Sinput}
> x <- round(x=runif(n=10, min=1, max=10), digits=0)
> x
\end{Sinput}
\begin{Soutput}
 [1]  4  4  3  9  6 10  2  9  9  7
\end{Soutput}
\begin{Sinput}
> sd(x)
\end{Sinput}
\begin{Soutput}
[1] 2.907844
\end{Soutput}
\end{Schunk}

\section{Kovarianz}
Die Kovarianz beschreibt ob sich zwei Datenreihen ähnlich entwickeln.
Dabei gilt, dass positive Werte der Kovarianz für ein ähnliches Verhalten
sprechen und negative dagegen. Der Betrag der Kovarianz gibt somit die 
Intensität der Aussage an. 
\[  
	s_{xy} 
	= \frac{1}{n-1} \cdot \left( 
		\sum_{i=1}^n (x_i - \bar{x}) \cdot (y_i - \bar{y})
	\right)
\]
Die Kovarianz lässt sich in \lstinline{R} mittels \lstinline{cov()}
ermitteln. Im Folgenden ein extremes positives Beispiel, generiert mit 
je 1000 sortierten exponentialverteilten Zufallszahlen. 
\begin{Schunk}
\begin{Sinput}
> x <- sort(x=rexp(n=1000))
> y <- sort(x=rexp(n=1000))
> cov(x,y)
\end{Sinput}
\begin{Soutput}
[1] 0.9493792
\end{Soutput}
\end{Schunk}

\section{Korrelation}

