\section{Aufgabe 2}
Die Zufallsvariable, die die Anzahl eingehender Telefonanrufe in einer 
Telefonzentrale innerhalb von 10 Minuten beschreibt, nennen wie $X$.
Sie folgt einer Poissonverteilung mit Erwartngswert $\lambda = 2$, d.h.
$X \sim$ Poisson($\lambda$).

\begin{enumerate}[(a)]
    \item Wie gross ist die Wahrscheinlichkeit, dass es in einer 
          bestimmten 10-Minuten-Periode keinen einzigen Anruf gibt?
    \item Wie gross ist die Wahrscheinlichkeit, dass es nicht mehr als drei 
          Telefonanrufe in einer bestimmten 10-Minuten-Periode gibt?
    \item Wie gross ist die Wahrscheinlichkeit, dass es mehr als drei 
          Telefonanrufe in einer bestimmten 10-Minuten-Periode gibt?
    \item Angenommen, die Anzahl Anrufe in einer 10-Minuten-Periode ist von 
          der Anzahl Anrufe in einer anderen 10-Minuten-Periode unabhängig. 
          Die Zufallsvariable, die die Anzahl Anrufe in einer Stunde 
          beschreibt bezeichnen wir mit Y . Welcher Verteilung folgt Y ?
\end{enumerate}

\subsection*{a)}
Es gilt zunächst mal $\lambda = 2 \rightarrow X \sim \text{Pois}(\lambda)$\\

\noindent
Hier kann wieder wie in Aufgabe 1 die Funktion \verb!dpois()! verwendet 
werden.
\begin{Schunk}
\begin{Sinput}
> dpois(0,2)
\end{Sinput}
\begin{Soutput}
[1] 0.1353353
\end{Soutput}
\end{Schunk}

\subsection*{b)}
Hier müssen wieder alle Wert für $0$ bis $3$ aufsummiert werden
\begin{Schunk}
\begin{Sinput}
> sum(dpois(0:3,2))
\end{Sinput}
\begin{Soutput}
[1] 0.8571235
\end{Soutput}
\end{Schunk}
oder man verwendet die Implementierung von R 
\begin{Schunk}
\begin{Sinput}
> ppois(3,2)
\end{Sinput}
\begin{Soutput}
[1] 0.8571235
\end{Soutput}
\end{Schunk}

\subsection*{c)}
Die Wahrscheinlichkeit, dass es in einer bestimmten 10-Minuten-Periode
mehr als drei Anrufe gibt, ist doch genau der Wert den man erhält, wenn
man die maximale Wahrscheinlichkeit ($\Omega = 1$) nimmt und den Wert für 
die Wahrscheinlichkeit von maximal $3$ Anrufen subtrahiert.
\begin{Schunk}
\begin{Sinput}
> 1 - sum(dpois(0:3,2))
\end{Sinput}
\begin{Soutput}
[1] 0.1428765
\end{Soutput}
\end{Schunk}
oder mit der vorbereiteten Funtkion
\begin{Schunk}
\begin{Sinput}
> 1 - ppois(3,2)
\end{Sinput}
\begin{Soutput}
[1] 0.1428765
\end{Soutput}
\end{Schunk}

\subsection*{d)}
Da die Ereignisse unabhängig sind, folgt $Y$ einer Poissonverteilung mit
einem $\lambda = 6 \cdot 2$ da es in einer Stunde 6 10-Minuten-Abschnitte gibt
und zu jedem dieser Abschnitte das ``Ursprungs-$\lambda$'' gilt.
\begin{Schunk}
\begin{Sinput}
> Wahrscheinlichkeit <- dpois(0:40,12)
> plot(Wahrscheinlichkeit, type='l')
\end{Sinput}
\end{Schunk}
\begin{figure}[h]
\begin{center}
\includegraphics{serie-5-2-fig1}
\caption{Poissonverteilung von $Y$ (Aufgabe 2-d)}
\end{center}
\end{figure}


