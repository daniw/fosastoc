\section{Aufgabe 1}

\subsection*{a)}
\[ H_0: \pi_0 = 0.15 \]
\[ H_A: \pi_A > 0.15 \]
\[ P_{H_0}(T \geq c) = 1- P(T \leq (c-1))\]

\begin{Schunk}
\begin{Sinput}
> 1-pbinom(0:16, size=16, prob=0.15)
\end{Sinput}
\begin{Soutput}
 [1] 9.257489e-01 7.160988e-01 4.386207e-01 2.101093e-01 7.905130e-02
 [6] 2.354438e-02 5.586261e-03 1.059004e-03 1.602100e-04 1.922270e-05
[11] 1.806631e-06 1.302180e-07 6.952283e-09 2.591237e-10 6.020962e-12
[16] 6.572520e-14 0.000000e+00
\end{Soutput}
\begin{Sinput}
> plot(1-pbinom(0:16, size=16, prob=0.15), type='l')
\end{Sinput}
\end{Schunk}
Hier sieht man, dass es ab dem 6. Glied das Signifikanzniveau unterschreitet
aber Achtung: R nummeriert diese Ergebnisse von 1 aus, d.h. das hier 
angeschriebene 6. Resultat ist eigentlich das 5.
\begin{Schunk}
\begin{Sinput}
> 1-pbinom(5,size=16,prob=0.15)
\end{Sinput}
\begin{Soutput}
[1] 0.02354438
\end{Soutput}
\end{Schunk}
Da wir aber $c-1$ berechnen mit dem, ist es halt das $(5+1)$. Glied, also das 
6. Glied.

\subsection*{b)}
Das Signifikanzniveau muss so gewählt werden, dass es unter dem 5. Glied 
unterschritten wird. Dies ist hier ca. 21\% also 0.2

\subsection{c)}
Die Frage ist eigentlich ``Wie gross ist die Wahrscheinlichkeit, dass
wir grösser das 6. Glied sind bei einer Ansprechwahscheinlichkeit von 30\%''.

Um dies zu berechnen summieren wir alle Wahrscheinlichkeiten ab dem 6. Glied.
\begin{Schunk}
\begin{Sinput}
> sum(dbinom(6:16, size=16, prob=0.30))
\end{Sinput}
\begin{Soutput}
[1] 0.3402177
\end{Soutput}
\end{Schunk}


\section{Aufgabe 3}

\subsection*{a)}
\begin{Schunk}
\begin{Sinput}
> binom.test(x=7,n=50,p=0.5,alternative="greater",conf.level=0.95)
\end{Sinput}
\begin{Soutput}
	Exact binomial test

data:  7 and 50 
number of successes = 7, number of trials = 50, p-value = 1
alternative hypothesis: true probability of success is greater than 0.5 
95 percent confidence interval:
 0.0675967 1.0000000 
sample estimates:
probability of success 
                  0.14 
\end{Soutput}
\begin{Sinput}
> binom.test(7,50)
\end{Sinput}
\begin{Soutput}
	Exact binomial test

data:  7 and 50 
number of successes = 7, number of trials = 50, p-value = 2.099e-07
alternative hypothesis: true probability of success is not equal to 0.5 
95 percent confidence interval:
 0.0581917 0.2673960 
sample estimates:
probability of success 
                  0.14 
\end{Soutput}
\end{Schunk}

\subsection*{b)}
Siehe a)
